\section{Feature Overview}

\noindent
\textbf{Repository:} \url{https://github.com/sigoden/aichat} \\
\textbf{Documentation:} \url{https://github.com/sigoden/aichat/wiki} \\
\textbf{Version:} 0.30.0 \quad \textbf{License:} MIT / Apache-2.0

\vspace{0.5cm}

\noindent
AIChat is a powerful and flexible LLM CLI tool designed to provide developers and advanced users with efficient and convenient AI interaction experiences.

\subsection{Core Features}

\begin{itemize}[leftmargin=*]
    \item \textbf{Multi-Provider Integration:} Support for 20+ LLM providers including OpenAI, Claude, Gemini, Ollama

    \textit{Usage:}
    \begin{verbatim}
    # Switch between different providers
    aichat -m gpt-4o "Explain quantum computing"
    aichat -m claude:claude-3-5-sonnet "Review my code"
    aichat -m ollama:llama3.1 "Local model query"
    \end{verbatim}

    \item \textbf{Flexible Working Modes:} Three distinct modes - CMD, REPL, and Server

    \textit{Usage:}
    \begin{verbatim}
    # CMD mode - quick one-shot queries
    aichat "What is Rust ownership?"

    # REPL mode - interactive conversations
    aichat
    > .help
    > .model gpt-4o

    # Server mode - API endpoints
    aichat --serve
    # Access at http://localhost:8000/playground
    \end{verbatim}

    \item \textbf{Shell Assistant:} Natural language to shell command conversion

    \textit{Usage:}
    \begin{verbatim}
    # Generate command without execution
    aichat -c "Find all Python files modified today"

    # Generate and execute command
    aichat -e "List top 5 largest files in current directory"

    # Complex operations
    aichat -e "Create backup of all .js files"
    \end{verbatim}

    \item \textbf{Rich Input Methods:} Support for files, directories, URLs, STDIN, and more

    \textit{Usage:}
    \begin{verbatim}
    # Single file
    aichat -f main.py "Explain this code"

    # Multiple files
    aichat -f src/*.rs "Find potential bugs"

    # Directory
    aichat -f ./docs/ "Summarize documentation"

    # URL
    aichat -f https://example.com "Summarize this page"

    # STDIN pipe
    cat error.log | aichat "Analyze these errors"

    # Combined inputs
    aichat -f config.yaml -f main.py "Check configuration"
    \end{verbatim}

    \item \textbf{Session Management:} Complete conversation context preservation and management

    \textit{Usage:}
    \begin{verbatim}
    # Start a new session
    aichat -s my-project "Let's design an API"

    # Continue existing session
    aichat -s my-project "Continue from yesterday"

    # List all sessions
    aichat --list-sessions

    # In REPL mode
    > .session my-project
    > .info session
    > .save session
    > .compress session
    \end{verbatim}

    \item \textbf{RAG Support:} Retrieval-Augmented Generation with external documents

    \textit{Usage:}
    \begin{verbatim}
    # Create RAG with documents
    aichat --rag project-docs
    > .rag add /path/to/docs/
    > .rag add README.md

    # Query your documents
    aichat --rag project-docs "How to deploy?"

    # Rebuild after document changes
    aichat --rag project-docs --rebuild-rag

    # Configure RAG settings
    # In config.yaml:
    # rag_embedding_model: openai:text-embedding-3-small
    # rag_top_k: 5
    \end{verbatim}

    \item \textbf{Function Calling:} Tool integration and automation capabilities

    \textit{Usage:}
    \begin{verbatim}
    # Enable function calling in config.yaml
    # function_calling: true
    # use_tools: 'fs,web'

    # Use with tools
    aichat "Search for latest Rust news and summarize"
    aichat "Read the config file and check for errors"

    # Custom tool mapping
    # mapping_tools:
    #   fs: 'fs_cat,fs_ls,fs_mkdir,fs_rm,fs_write'
    #   web: 'web_search,web_fetch'
    \end{verbatim}

    \item \textbf{AI Agents:} Complex task automation and workflow execution

    \textit{Usage:}
    \begin{verbatim}
    # List available agents
    aichat --list-agents

    # Start an agent
    aichat -a code-reviewer "Review my code"

    # Agent with variables
    aichat -a code-reviewer \
      --agent-variable language rust \
      -f main.rs "Review this file"

    # Create custom agent
    # ~/.config/aichat/agents/my-agent/config.yaml
    # name: my-agent
    # model: gpt-4o
    # use_tools: 'fs,web'
    # instructions: |
    #   You are a specialized assistant...
    \end{verbatim}
\end{itemize}

\subsection{Additional Features}

\begin{itemize}[leftmargin=*]
    \item \textbf{Role System:} Customize AI behavior with predefined roles
    \begin{verbatim}
    aichat -r code "Write a web server"
    aichat -r translator "Translate: Hello World"
    \end{verbatim}

    \item \textbf{Macro System:} Combine commands into reusable shortcuts
    \begin{verbatim}
    aichat --macro review main.rs
    aichat --list-macros
    \end{verbatim}

    \item \textbf{Custom Themes:} Personalize your terminal experience
    \begin{verbatim}
    # config.yaml
    # highlight: true
    # light_theme: false
    \end{verbatim}
\end{itemize}

\subsection{Quick Start}

\begin{verbatim}
# Installation
cargo install aichat         # Rust
brew install aichat          # macOS/Linux
pacman -S aichat            # Arch Linux

# Configuration
# Edit ~/.config/aichat/config.yaml
# Add your API keys:
# clients:
#   - type: openai
#     api_key: sk-xxx

# Basic usage
aichat "Hello, AIChat!"

# Interactive mode
aichat

# Get help
aichat --help
\end{verbatim}
